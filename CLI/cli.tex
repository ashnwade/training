\chapter{Command Line Interface}
\index{Command-line interface}
In addition to the GUI we've shown so far, Gravwell provides a
command-line client. This client can be useful for certain repetitive
tasks such as testing scripts (see the Automation chapter for more info)
or in situations where the GUI is not available. It can also be used in
shell scripts or cron jobs to automatically gather information from a
Gravwell instance, if needed. The intent is that the command line
interface should provide every function available in the GUI.

\section{Running the Client}

The command-line client is installed as \code{/usr/sbin/gravwell} by
default. It will attempt to connect to localhost unless a different
server is specified with the ``-s'' flag, but be aware that since Gravwell's
default install disables HTTPS, we'll need to use the ``-insecure-no-https''
flag. The client will prompt for a username and password; use the same
credentials you would use to log in to the GUI:

\begin{Verbatim}[breaklines=true]
/ # gravwell -insecure-no-https
Username:  admin
Password:  changeme
#>
\end{Verbatim}

Once you log in, the client will show a top-level prompt. Similar to
the Cisco command line, the client implements multi-level menus; by
typing `admin', we can enter the administrative menu, then we can type
`help' to get a listing of available commands. Here is a snippet of
available commands:

\begin{Verbatim}[breaklines=true]
#>  admin
admin>  help
add_user            Add a new user
impersonate_user    Impersonate an existing users
del_user            Delete an existing user
get_user            Get an existing users details
update_user         Update an existing user
list_users          List all users
lock_user           Lock a user account
user_activity       Show a specific users activity
\end{Verbatim}

Help is available at all menu levels. When a command is selected,
the client will prompt for input:

\begin{Verbatim}[breaklines=true]
#>  search
query>  tag=* count
time range> -6h
count 1100
1/1
Press q[Enter] to quit, [Enter] to continue

Total Items: 1
1101 stats records from Mar 22 16:29:53.000 <-> Mar 22 22:29:54.000
count 1.10 K/1.00 663.35 KB/547 B 65.238537ms
\end{Verbatim}

\section{Hands-on Lab: Basic CLI exploration}

Launch a Gravwell webserver+indexer container:

\begin{Verbatim}[breaklines=true]
docker run --rm --net gravnet -p 8080:80 -d --name gravwell gravwell:base
\end{Verbatim}

Now get a shell on the Gravwell container and run the client:

\begin{Verbatim}[breaklines=true]
$ docker exec -it gravwell /bin/sh
/ # gravwell -insecure-no-https
Username:  admin
Password:  changeme
#>
\end{Verbatim}

Try the following tasks:

\begin{enumerate}
\item
  {Create a new user}
\item
  {Lock that user}
\item
  {Verify that the user has been locked.}
\item
  {Unlock the user}
\item
  {Delete the user}
\end{enumerate}

When you're done, use the ``exit'' command to quit the CLI, then ``exit'' again to leave the shell in the Docker container.

Now, let's use the command line client non-interactively to run
a search and download the results. First, we'll ingest 1,000 random JSON
entries into the Gravwell instance:


\begin{Verbatim}[breaklines=true]
docker run --net gravnet --rm -it --name ingesters gravwell:ingesters \
/opt/gravwell/bin/jsonGenerator -clear-conns gravwell:4023 -entry-count 1000
\end{Verbatim}


Now, we use the client to run a search in the background; the search
will find all entries containing the name ``David'':

\begin{Verbatim}[breaklines=true]
$ docker exec -it gravwell /bin/sh
/ # gravwell -insecure-no-https -b search
query>  tag=json grep David
time range> -6h
Background search with ID 569547375 launched
\end{Verbatim}

And use the client to download the results:

\begin{Verbatim}[breaklines=true]
/ # gravwell -insecure-no-https download
+----------+-----+---------+----------+--------------------+-----------+
|Search ID |User |   Group |    State |   Attached Clients |   Storage |
+==========+=====+=========+==========+====================+===========+
|569547375 |   1 |       0 |  DORMANT |                  0 |  12.66 KB |
+----------+-----+---------+----------+--------------------+-----------+
search ID>  569547375
Available formats:
json
text
csv
format>  text
Save Location (default: /tmp)>  /tmp
Saving to  /tmp/569547375.text
\end{Verbatim}

\subsubsection{Lab Questions}

\begin{enumerate}
\item
  Download the results in csv and json format. What's the difference?
\item
  What happens when you leave out the ``-b'' flag while launching the
  search?
\end{enumerate}

To clean up after the experiment, run:

\begin{Verbatim}[breaklines=true]
docker kill $(docker ps -a -q)
\end{Verbatim}
