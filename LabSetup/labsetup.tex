\chapter{Lab Setup and Docker Testing}

We will be making extensive use of the Docker container platform during
the hands on exercises. The purpose of this chapter is to ensure that
all training attendees have a Linux system with a functioning Docker
installation. To fully participate in the training each attendee will
need the following:

\begin{enumerate}
	\item Workstation running AMD64 Linux or a virtual machine running 64-bit Linux
	\begin{enumerate}
		\item At least 4GB of available RAM
		\item At least 10GB disk storage available for Docker instances
	\end{enumerate}
	\item Ethernet adapter and cable with connectivity to training server
	\item Standards-compliant web browser (Firefox, Chrome, Chromium)
\end{enumerate}

\section{Verifying Docker Installation}

Docker is available in most Linux package managers on modern Linux
distributions. Most Debian based distributions can install Docker via
the command \code{apt install docker.io} and Redhat based distributions can
install docker using \code{yum install docker}. Make sure to execute these
commands as the super user, either via \code{sudo} or by first becoming
root using \code{su}.

\section{Getting Started With Docker}

The hands on training segments make heavy use of
Docker\footnote{https://www.docker.com}. Docker is
a container platform designed to simplify the deployment and management
of microservices. While Gravwell is not a microservice, Docker provides
an easy to use and clean environment to experiment with Gravwell
services. This section will ensure that you have a working Docker
environment and can import Gravwell images.

\subsection{Host System Requirements}

The first thing to verify is that the system that will be hosting
Docker and our gravwell images is a modern 64bit Linux kernel. We need
at least kernel version 3.2 running on an x86 architecture with 64bit
extensions. Any X86 AMD or Intel processor manufactured within the last
10 years should work. Open a command prompt and run the \code{uname
-a} command. We want to see \code{x86\_64} and a kernel version greater
than 3.2.0:

\code{Linux training 4.18.0-16-generic \#17-Ubuntu SMP Fri Feb 8 00:06:57 UTC
2019 x86\_64 x86\_64 x86\_64 GNU/Linux}


\subsection{Granting User Access to Docker}

Docker requires elevated privileges to start containers, build
networks, and generally manage the Docker process. If you don't mind
interacting with Docker as the superuser you can skip this section, but
if you would like to be able to run the labs as a non-privileged user
(recommended) we will set up the \code{docker} group and add your current
user to it. This will allow your current user to drive Docker.

First check if your user can run Docker commands by executing \code{docker
ps -q} and \code{docker network ls} as a non-root user. If the output is
empty or contains a list of running Docker containers, your user already
has access to the Docker API. If errors were displayed, we will need to
create the docker group and add your current user to it. This is done
using the \code{groupadd} and \code{usermod} commands. First run \code{groupadd
docker} as the superuser, then run \code{usermod -aG docker
<username>} where
\code{<username>} is the user you wish to use for
the training labs. You will need to logout of the current user and then
log back in so that the group permission changes can propagate. You can
also just reboot your machine/VM. If all goes well, you should be able
to run the docker commands as a non-root user.

\section{Testing Docker Image Imports}

A core component of the hands-on training labs is importing prebuilt
Gravwell images for experimentation. We need to walk through the basic
Docker commands to import and/or load a Docker image and run it. You
should have been provided with a \code{tar.gz} archive containing materials for
the course. Make sure it's unpacked, then look in the
\code{dockerimages} subdirectory for a file named ``\code{test.tar.gz}''. Load it
into your local docker image repository using the command \code{docker load
-i test.tar.gz}, then verify that the image was imported using the
\code{docker images} command. You should see a new image tagged (named) \code{gravwell:test}.

Now we can start up the test container using Docker and poke around a
little. A container is like a very lightweight virtual machine. Note that a
container is not quite the same as a Type 1 or Type 2 virtual machine,
since it uses the same kernel as the host system.\footnote{https://www.backblaze.com/blog/vm-vs-containers/}.

\section{Starting a Container in Docker}

Now that we have imported a Docker image, let's create a new container
and poke around in it a bit. We are going to start the
\code{gravwell:test} image and execute the \code{/bin/sh} command in
interactive mode. This means that we will immediately drop into the
container. To keep things clean we will also pass the
\code{-\/-rm} flag which tells Docker to destroy the container when the
main process (\code{/bin/sh}) exits. Start the container and poke around a
bit, check the IPs, proc filesystem, and running processes. The
container \emph{almost} looks like a real machine.

\begin{Verbatim}[breaklines=true]
user@training:~$ docker run -it --rm gravwell:test /bin/sh
# ps
PID   USER     TIME  COMMAND
    1 root      0:00 /bin/sh
    6 root      0:00 ps
# ip addr
1: lo: <LOOPBACK,UP,LOWER_UP> mtu 65536 qdisc
noqueue qlen 1000
    link/loopback 00:00:00:00:00:00 brd 00:00:00:00:00:00
    inet 127.0.0.1/8 scope host lo
       valid_lft forever preferred_lft forever
7: eth0@if8:
<BROADCAST,MULTICAST,UP,LOWER_UP,M-DOWN qdisc
noqueue
    link/ether 02:42:ac:11:00:02 brd ff:ff:ff:ff:ff:ff
    inet 172.17.0.2/16 brd 172.17.255.255 scope global eth0
       valid_lft forever preferred_lft forever
# whoami
root
# pwd
/
# exit
user@training:~$
\end{Verbatim}

Type exit to leave the container, and run \code{docker ps -a} to list all
active containers. Because we gave the \code{-\/-rm} flag the container
should be completely gone.

WARNING: Do not use the \code{-\/-rm} flag on containers that you want
to keep or that you expect to continue to work on; one accidental
\code{exit} command will completely destroy all your work.

\section{Testing Gravwell in a Container}

Let's fire up a minimal Gravwell instance in a docker container and
verify that we can get to the web portal. A minimal Gravwell instance
is available in the images directory at \code{dockerimages/base.tar.gz} within
the training directory. Load it into your local images repository,
then run it without any arguments. The base image already has
configuration parameters which specify which processes should be started
within the container. We will be naming this container \code{base} using
the \code{-\/-name} parameter and also starting the container in the
background using the \code{-d} parameter.

\begin{Verbatim}[breaklines=true]
~/gravwell_training$ docker load -i dockerimages/base.tar.gz 
Loaded image: gravwell:base
~/gravwell_training$ docker run -d --name test gravwell:base
7c18342e7dcc4e362323797954bd1b1742dd903ded097331f5c92e13853bd628
~/gravwell_training$ docker ps -q
7c18342e7dcc
\end{Verbatim}

Verify that the container is running using docker ps, then let's pull
out the IP that Docker assigned our container so that we can hit the
Gravwell webserver.

\code{docker inspect test}

Visit the IP address in your browser; you should see the Gravwell web page.
Log into Gravwell with the credentials admin/changeme to make sure everything is OK,
then go back to your command prompt and stop and clean up the base image.

\begin{Verbatim}[breaklines=true]
docker stop test
docker rm test
\end{Verbatim}

\section{Loading the Lab Images}
\label{sec:load-lab-images}

Several different Docker container images are used for the lab sections
of this training. They are available in the \code{dockerimages/} subdirectory on
the course materials bundle and can be installed with the following
command:

\code{docker load -i imagename.tar.gz}

Fetch and load the following images from the dockerimagesimages folder:

\begin{itemize}
	\item base.tar.gz
	\item cloudarchive\_server.tar.gz
	\item datastore.tar.gz
	\item indexer.tar.gz
	\item ingesters.tar.gz
	\item offlinereplication.tar.gz
	\item webserver.tar.gz
\end{itemize}

Once loaded, they should all be in the listing when you run \code{docker
image ls}.

\section{Creating the Gravwell Test Network}

To help keep our Gravwell experimental containers separate, we put them
all on a Docker network called ``gravnet''. This is also what allows
containers to connect to each other by name rather than by IP, which is
an important part of our experimental config. Before you use the
network, you must create it:

\code{docker network create gravnet}


\section{Helpful Docker Tips}

This section has some commands or scripts to help you when doing this
training.

To show all IPs and names:

\begin{Verbatim}[breaklines=true]
docker ps -q | xargs -n 1 docker inspect --format \
'{{range .NetworkSettings.Networks}}{{.IPAddress}}{{end}} {{ .Name }}' \
| sed 's/ \// /'
\end{Verbatim}

To kill all running containers:

\code{docker kill \$(docker ps -q)}

