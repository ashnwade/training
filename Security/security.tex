\chapter{Securing Gravwell}
\index{Security}
This chapter discusses security considerations for a Gravwell cluster.
It will discuss indexer and webserver security, securing ingesters, and
user authentication.

This chapter makes frequent references to \textbf{shared secrets}. These are
strings used by Gravwell components to communicate with each other
without user intervention. Shared secrets are never transmitted directly
over the network; instead, clients and servers use a challenge-response
algorithm to verify that they have the same shared secret. Because the
shared secrets must be stored in plaintext to perform
challenge-response, they are potentially vulnerable to users on the same
machine and should therefore be protected. The shared secrets can be
stored in three different ways:

\begin{enumerate}
\tightlist
\item
  Directly in \code{gravwell.conf}. This is the simplest, but care should
  be taken to ensure the file is only readable by the gravwell user.
\item
  As environment variables. If done correctly, this can help protect
  the secrets, but take note that setting the environment variables in a
  Docker command line (e.g. \code{`-e GRAVWELL\_INGEST\_AUTH=MySecret'}) can
  make them vulnerable to interception by other users. See
  the Gravwell online documentation\footnote{https://docs.gravwell.io/\#!configuration/environment-variables.md} for a description of environment variables.
\item
  As a Docker secret. This is only available for systems running in
  Docker containers. See
  the Gravwell online documentation\footnote{https://docs.gravwell.io/\#!configuration/environment-variables.md} for a description of how to pass Docker secret files to a Gravwell instance.
\end{enumerate}



\section{TLS/HTTPS}
\label{sec:tls}
\index{Security!encryption}\index{Security!TLS}\index{Webservers!security}
By default, Gravwell is configured to communicate via unencrypted
channels. This means that communications between the webserver and the
user, as well as communication between ingesters and indexers, will be
visible to any adversary capable of intercepting traffic. As later
sections will explain, Gravwell's built-in authentication methods ensure
that the ingester's shared secrets will not be leaked, but an adversary
could potentially capture passwords as users authenticate to the
webserver, view entries as they are ingested, etc.

Switching over to TLS-encrypted communications will have two benefits:
it will encrypt traffic to prevent eavesdropping, and it will allow
components to verify that they are actually connecting to the correct
system rather than an adversary's man-in-the-middle attack. The
datastore component cannot be used without installing TLS certificates,
so enabling TLS encryption is also a necessary part of configuring
distributed frontends.

We strongly recommend installing a properly-signed TLS certificate on
the Gravwell webserver and indexer before deploying Gravwell in any
production environment. If a properly-signed certificate is unavailable,
Gravwell can be deployed with a self-signed certificate, but every
component will need to be configured to ignore invalid certificates,
which makes the system vulnerable to man-in-the-middle attacks.

\subsection{Installing a properly-signed TLS certificate}

A properly-signed TLS certificate is the most secure way to access
Gravwell. Browsers will automatically accept the certificate without
complaint.

Obtaining a certificate is outside the scope of this documentation;
consider either purchasing a certificate through one of the traditional
providers or
using LetsEncrypt\footnote{https://letsencrypt.org} to
obtain a free one.

To use the certificate, Gravwell must be told where the certificate and
key files are. Assuming the files are installed at \code{/etc/certs/cert.pem} and
\code{/etc/certs/key.pem}, edit \code{gravwell.conf} to uncomment and populate
the \code{Certificate-File} and \code{Key-File} options:

\begin{Verbatim}[breaklines=true]
Certificate-File=/etc/certs/cert.pem
Key-File=/etc/certs/key.pem
\end{Verbatim}

Note: These files must be readable by the ``gravwell'' user, but make
sure other users can't read them.

To enable HTTPS on the webserver, change the \code{Web-Port} directive
from 80 to 443, then comment out the
\code{Insecure-Disable-HTTPS} directive. To enable TLS-encrypted ingester
connections, find and uncomment the line \code{TLS-Ingest-Port=4024}. To
enable HTTPS for the search agent, open
\code{/opt/gravwell/etc/searchagent.conf}, comment out the
\code{Insecure-Use-HTTP=true} line and change the port in the
\code{Webserver-Address} line from 80 to 443.

Finally, restart the webserver, indexer, and search agent:

\begin{Verbatim}[breaklines=true]
systemctl restart gravwell_webserver.service
systemctl restart gravwell_indexer.service
systemctl restart gravwell_searchagent.service
\end{Verbatim}

\subsection{Install a self-signed certificate}

Although it is not as secure as a proper TLS certificate, a self-signed
certificate will ensure encrypted communication between users and
Gravwell. By instructing browsers to trust the self-signed cert, it is
also possible to avoid recurring warning screens.

First, we will generate a 1-year certificate in
\code{/opt/gravwell/etc} using \code{gencert}, a program we ship with the
Gravwell install:}

\begin{Verbatim}[breaklines=true]
cd /opt/gravwell/etc
sudo -u gravwell ../bin/gencert -h HOSTNAME
\end{Verbatim}

Make sure to replace HOSTNAME with either the hostname or the IP
address of your Gravwell system. You can specify multiple hostnames or
IPs by separating them with commas, e.g. \code{gencert -h gravwell.example.com,10.0.0.1,192.168.0.3}

Now, open \code{gravwell.conf} and uncomment the \code{Certificate-File} and
\code{Key-File} directives. The defaults should point correctly to the two
files we just created.

To enable HTTPS on the webserver, change the \code{Web-Port} directive
from 80 to 443, then comment out the
\code{Insecure-Disable-HTTPS} line. To enable TLS-encrypted ingester
connections, find and uncomment the line \code{TLS-Ingest-Port=4024}.

To enable HTTPS for the search agent, open
\code{/opt/gravwell/etc/searchagent.conf}, comment out the
\code{Insecure-Use-HTTP=true} line and change the port in the
\code{Webserver-Address} line from 80 to 443. With a self-signed
certificate, you will also need to ensure that \code{Insecure-Skip-TLS-Verify=true} is set.

Finally, restart the webserver, indexer, and search agent:}

\begin{Verbatim}[breaklines=true]
systemctl restart gravwell_webserver.service
systemctl restart gravwell_indexer.service
systemctl restart gravwell_searchagent.service
\end{Verbatim}

\section{Indexer Security}
\index{Indexers!security}
Gravwell indexers communicate with two components: ingesters, and
webservers. Users never interact directly with indexers.

\subsection{Authentication \& Secrets}

Ingesters and webservers authenticate to the indexer using shared
secrets. Ingesters are authenticated using the \code{Ingest-Auth} secret
while webservers use the \code{Control-Auth} secret. These shared secrets
are stored in plaintext and used to perform challenge-response
authentication. This means ingesters and webservers can auth to the
indexer over plaintext channels without leaking the shared secrets, but
it also means these secrets are vulnerable to attackers who have access
to the filesystems of these systems.

\subsection{Indexer-Webserver Communications}

Communications between the indexer and the webserver take place over
TCP port 9404 by default (specified by the \code{Control-Port} parameter).
These connections \emph{are not encrypted}. Indexers often have to transfer
very large numbers of entries to the webserver during a search, and
encryption would significantly slow the transfer. For this reason,
indexers and webservers should communicate only over trusted networks.
If the indexer-webserver traffic must traverse the Internet, we
strongly recommend using a VPN or other tunnel to transport the
traffic.

\subsection{Indexer-Ingester Communications}

Ingesters can connect to indexers using either encrypted or unencrypted
connections. By default, indexers listen for cleartext ingest
connections on TCP port 4023. If a TLS certificate is set up on the
indexer, it can also listen for encrypted connections; by convention,
indexers should listen for TLS ingest connections on TCP port 4024.


\section{Webserver Security}
\index{Webservers!security}
Webservers communicate with user clients, indexers, the search agent,
and optionally the datastore. As the gateway into the Gravwell system,
they are often accessible from public networks and should therefore be
protected carefully.

\subsection{Authentication \& Secrets}

Webservers store the following shared secrets:

\begin{itemize}
\tightlist
\item
  \code{Control-Auth}, used to authenticate the webserver to indexers and
  the datastore.
\item
  \code{Search-Agent-Auth}, used by the search agent to authenticate to the
  webserver.
\end{itemize}

They also maintain a database of Gravwell users, stored by default in
\code{/opt/gravwell/etc/users.db}. Passwords in the user database are
bcrypt hashed, but care should still be taken to prevent unauthorized
system users from reading this file.

\subsection{Webserver-Indexer Communication}

Communications between the indexer and the webserver take place over
TCP port 9404 by default (specified by the \code{Control-Port} parameter).
These connections \emph{are not encrypted}. Indexers often have to transfer
very large numbers of entries to the webserver during a search, and
encryption would significantly slow the transfer. For this reason,
indexers and webservers should communicate only over trusted networks.
If the indexer-webserver traffic must traverse the Internet, we
strongly recommend using a VPN or other tunnel to transport the
traffic.

\subsection{Webserver-User Communication}

Users access the webserver via their web browsers over either HTTP or
HTTPS. Authentication consists of an HTTP POST request containing a
username and password, meaning a user's password may be transmitted in
plaintext if TLS is not configured on the webserver. Because of this, \emph{no
Gravwell webserver should be used over the Internet without first
installing TLS certificates.}

\subsection{Webserver-Search Agent Communication}

The automated search agent connects to a Gravwell webserver over the
same HTTP or HTTPS ports as a regular user, but it authenticates using a
challenge-response algorithm and the search agent-specific shared
secret. The search agent will default to unencrypted HTTP, which may be
acceptable when the search agent is on the same system as the webserver,
but if a search agent is installed on a separate system it should be
configured to use HTTPS.

\subsection{Webserver-Datastore Communication}

When multiple webservers are configured in a Gravwell cluster, they
coordinate via the datastore component. Communications between the
webservers and the datastore include (hashed) user passwords, search
histories, user-uploaded resources, and other potentially sensitive
items. For this reason, Gravwell \emph{strongly suggests} that communication between
the webserver and the datastore be TLS-encrypted; configuring a TLS
certificate as described in an earlier section is an important step in
setting up a Gravwell cluster with multiple webservers.


\section{Ingester security}
\index{Ingesters!security}
Ingesters communicate only with indexers, using a shared ingest secret
which can be stored via the methods described at the beginning of this
chapter. As documented in the section on indexer security, ingesters may
communicate with the indexers over either cleartext or encrypted
connections. In either case, the shared secret is never transmitted, so
it is not susceptible to interception, but an adversary could
potentially intercept \emph{entries} as they are uploaded over a cleartext
connection. For this reason we strongly recommend configuring TLS
certificates on the indexer if you intend to run ingesters outside your
trusted networks.

If all ingesters use the same ingest secret, and one is compromised, the attacker
could theoretically use that compromised ingest secret to upload entries to
your indexers. Consider deploying Federators (see Section \ref{sec:federator}) in
order to segment your ingesters into different security enclaves to help mitigate this risk.

\section{User Authentication and Lockout}
\index{Users}
As mentioned in the webserver section, users are authenticated to the
webserver by sending their username and password in an HTTP POST
request. If the webserver is not configured to offer HTTPS connections,
user credentials are at risk of interception. Always set up TLS
certificates on the webserver and enable HTTPS before allowing users to
log on from public networks.

Gravwell provides a basic form of brute-force protection. If several
failed login attempts are made over a short period, the webserver will
delay its response to subsequent attempts by several seconds. This is
intended to slow down brute-forcers while still allowing legitimate
users to log in.
